\documentclass[11pt,twoside]{memoir}


\usepackage{packages/setup}
\usepackage{packages/bibconfig}
%loads bibliography package for this document
\addbibresource{einfalg/lit.bib}
\usepackage{packages/styles}

\usepackage{packages/theorems}
\usepackage{packages/mymathtools}


\title{Einführung in die Algebra}
\pretitle{\pagestyle{empty}\begin{center}  \Huge \bfseries}
	\renewcommand{\maketitlehookb}{\centering \Large  \textit{Vorlesungsmitschriften im Wintersemester 2018/19}}
\date{}

\makeatletter
\newlength{\@thlabel@width}%
\newcommand{\thmenumhspace}{\settowidth{\@thlabel@width}{\itshape1.}\sbox{\@labels}{\unhbox\@labels\hspace{\dimexpr-\leftmargin+\labelsep+\@thlabel@width-\itemindent}}}
\makeatother

\begin{document}
	
\frontmatter
\maketitle
\begin{KeepFromToc}
	\tableofcontents
\end{KeepFromToc}
\chapter*{Vorwort}
Diese Vorlesungsmitschriften werden in der Vorlesung \textit{Einführung in die Algebra} von Prof. Jan Schröer im Wintersemester 2018/19 an der Universität Bonn angefertigt.\par
Wir versuchen, diese immer unter \url{https://pankratius.github.io} zu aktualisieren.

I will write them in English, as Prof. Schroer already provides a german version of his lecture notes. In addition, the first two lectures are ommited, as they were only motivational, but my motivation to draw a lot of pictures is fairly limited.

\mainmatter
\chapter{Gruppen}

\section{Definitionen - Grundlegendes}
\lipsum[1-5]
\section{Abbildungen von Gruppen}
\subsec{Gruppenhomomorphismen}
\lipsum[1]
\subsec{Gruppenisomorphismen}
\begin{lem}\label{thm:Apfel}
	Ich bin ein Apfel.
\end{lem}
\lec

\begin{defn}
	Ein \toidx{Gruppen\isom} ist ein Isomorphismus in der Kategorie der Gruppen. \Cref{thm:Apfel}
\end{defn}
\newpage
\section{Normalteiler und Quotientengruppen}
\begin{prop}\label{trialprop}
	Normalteiler sind ziemlich cool.
	\[
	\begin{tikzcd}
	G \ar{r}{\phi} \ar[swap]{d}{\pi} & H\\
	G/N \ar[swap]{ur}{\exists! f}
	\end{tikzcd}
	\]
	Wir erhalten für den Fall, dass $\phi$ surjektiv ist, folgendes Bild:

	\[
	\begin{tikzcd}
	G \ar[\surarr]{r}{\pi}  \ar[bend left = 25]{rrr}{\phi}& G/\ker\phi \ar{r}{\cong} & \im \phi \ar[\embarr]{r}{} & H
	\end{tikzcd}
	\]
\end{prop}
\lec
\begin{defn}
	Ein kommutatives Diagramm
	\[
	\begin{tikzcd}
	0\ar{r}&M_1\ar{r}&N\ar{r}&M_2\ar{r}&0
	\end{tikzcd}
	\]
	heißt \toidx{split}, falls es isomorph zu einem Diagramm der Form
	\[
	\begin{tikzcd}
	0\ar{r}&M_1\ar{r}\ar{d}{\sim}&N\ar{r}\ar{d}{\sim}&M_2\ar{r}\ar{d}{\sim}&0\\
	0\ar{r}&M_1'\ar{r}&M_1'\oplus M_2'\ar{r}&M_2'\ar{r}&0
	\end{tikzcd}
	\]
	ist.
\end{defn}
\begin{prop}
	Sei $f\colon M\to N$ ein Homomorphismus zwischen $A$-Moduln $M,N$. Dann hat $f$ ein linksinverses, genau dann wenn das Diagramm
	\[
	\begin{tikzcd}
	0\ar{r}& M \ar{r}{f}& N\ar{r}&\mathrm{Coker}(f)\ar{r}&0
	\end{tikzcd}
	\]
	split ist.
\end{prop}
\begin{proof}
	Angenommen, $f$ hat ein linksinverses, so dass 
	\[
	\begin{tikzcd}
	0\ar{r}&M\ar{r}{f}\ar[swap,equal]{dr}&N\ar{d}{\psi}\\
			&&M
	\end{tikzcd}
	\]
	kommutiert.
\end{proof}
\begin{thm}[Zerlegungssatz für regülare Metriken]
	Sei $(V,s)$ ein regulärer symmetrischer $K$-Vektorraum, und $U\subset V$ ein Untervektorraum. Dann gilt
	\[\dim(V) = \dim(U) + \dim(U^{\perp})\]
	Ist $s_u$ regulär, so gilt außerdem 
	\begin{enumerate}
		\item $V = V \perp U^{\perp}$
		\item $s_{U^{\perp}}$ ist regulär
	\end{enumerate}
\end{thm}
Wir betrachten nun Isometrien eines metrischen Vektorraums. Dafür definieren wir die \toidx{Isometriegruppe} eines metrischen Vektorraumes $(V,s)$ über dem Körper $K$ mit Involution $\overline{\cdot}$ als
\[
\gl(V,s):= \{f:V\to V\ssp f~\text{ist eine Isometrie}\}
\]
Für isometrische Vektorräume gelten folgenden äquivalente Bedingungen
\begin{prop}
	Seien $(V,s_V)$ und $(W,s_W)$ endlich-dimensionale metrische $K$-Vektorräume bezüglich der Involution $\overline{\cdot}$. Dann sind äquivalent:
	\begin{enumerate}
		\item $(V,s_V)\cong (W,s_W)$.
		\item Es gibt Basen $B$ bzw. $C$ von $V$ bzw. $W$, so dass $\mathbf{c}_B(s_V) = \mathbf{c}_C(s_W)$.
		\item Für alle Basen $B$ bzw. $C$ von $V$ bzw. $W$, ist $\mathbf{c}_B(s_V) \equiv \mathbf{c}_C(s_W)$.
	\end{enumerate}
\end{prop}
\chapter{Homological Algebra}
\section{Cat. Prelim}
\subsection{Additive Categories}
\begin{defn}
	Let $M,N$ be objects in a category $\cat{C}$. A \morph $\phi\in \hom_{\cat{C}}(M,N)$ is called a \toidx{monomorphism}, if for all objects $Z$ in $\cat{C}$ and morphisms $\zeta_1,\zeta_2\in \hom_{\cat{C}}(Z,M)$, $\zeta_1\circ\phi=\zeta_2\circ \phi\implies \zeta_1=\zeta_2$,
	\[
	\begin{tikzcd}
	Z\ar[shift left]{r}{\zeta_1}\ar[shift right,swap]{r}{\zeta_2}&M\ar{r}{\phi}&N.	\end{tikzcd}
	\]
\end{defn}	
\begin{lem}
	Let $\cat{A}$ be an additive category. For all $M,N$ objects in $\cat{A}$ and morphisms $\phi \in \hom_{\cat{A}}(M,N)$ the following are equivalent:
	\begin{enumerate}
		\item $\phi$ is a monomorphism.
		\item For all objects $Z$ in $\cat{A}$ and $\zeta \in \homc{C}(Z,A)$, $\zeta \circ \phi \implies \phi = 0$.
	\end{enumerate}
\end{lem}
\subsection{Abelian Categories}
\begin{defn}
	An additive category $\cat{A}$ is called \toidx{abelian} if kernels and cokernels exist in $\cat{A}$; every monomorphism is the kernel of some morphism and every epimorphism is the cokernel of some morphism.
\end{defn}
\begin{prop}
	$R-\cmod$ is an abelian category.
\end{prop} 
\begin{proof}
	We will show that every mono\morph is the \tker of some mono\morph. Let $\phi:M\to N$ be a mono\morph; hence $\phi$ is injective. Consider now the projection $\pi: N \msurarr \coker\phi$. Then, by the first isomorphism theorem (\cref{trialprop}, $\ker \pi = \im \phi \cong M$ and the inclusion map $\ker \pi \membarr N$ is injective.\\
	Let $Z$ be another $R$-module and $\zeta:Z\to M$ a morphism such that $\pi \circ \zeta = 0$. Consider the following commutative diagram
	\[
	\begin{tikzcd}
	Z\ar{r}{\zeta}\ar[dashed]{dr}\ar[dashed, bend left = -10]{ddr} \ar[bend left=25]{rr}{0} & N \ar{r}{\pi}& \coker \phi \\
	& \ker \pi \ar[hookrightarrow]{u}& \\
	& M \ar[leftrightarrow]{u}{\cong} \ar[bend left= -35,swap]{uu}{\phi}&
	\end{tikzcd}
	\]
	Hence $\phi:M\to N$ is a kernel of $\coker \phi$.
\end{proof}
This notion continues to hold in every abelian category.
\begin{prop}
	In an abelian category $\cat{A}$, every kernel is the kernel of its cokernel; every cokernel is the cokernel of its \tker.
\end{prop}
\begin{proof}
	Let $\phi:A\to B$ be the \tcker of some morphism $Z\to A$, and let $\iota: K\to A$ be the kernel of $\phi$. Since $Z\to A \to B$ is zero, $Z\to A$ factors uniquely to through $\iota$, by the definition of the kernel:
	\[
	\begin{tikzcd}
		Z\ar{r}\ar[dashed]{dr} & A \ar{r}{\phi} &B\\
		&K \ar{u}{\iota}&
	\end{tikzcd}
	\]
	Now let $A \to C$ be any morphism such that $K\to A\to C$ is the zero-\morph. Then the induced morphism $Z\to A \to C$ is also zero, and hence factors through $\phi$, since $\phi$ is the cokernel of $Z\to A$:
	\[
		\begin{tikzcd}
		& C&\\
		Z\ar{r}\ar[dashed]{dr} \ar{ur} & A\ar{u} \ar{r}{\phi} &B\ar[dashed]{ul}\\
		&K \ar{u}{\iota}&
		\end{tikzcd}
	\]
	But this implies that every morphism $A\to C$, such that $K\to A\to C = 0$, factors uniquely through $A\to B$. Hence the cokernel $A\to B$ is the cokernel of its kernel $K\to A$.\par
	As exercise, we show the dual part as well. Let $\phi: A\to B$ be the kernel of some morphism $B\to Z$, and $\pi: B \to C$ be the cokernel of $\phi$. Since $A\to B \to Z = 0$, $B \to Z$ factors uniquely through the cokernel $B\to C$ of $\phi$:
	\[
	\begin{tikzcd}
	&C\ar[dashed]{dr}&\\
	A\ar{r}{\phi}&B \ar{u}\ar{r}& Z
	\end{tikzcd}
	\]
	Now let $K \to B$ be any morphism such that the composition $K\to B \to C$ is zero. Then the induced map $A\to B\to Z$ is also zero, and hence factors uniquely through $\phi:A\to B$, as $\phi$ is the kernel of $B\to Z$:
		\[
		\begin{tikzcd}
		&C\ar[dashed]{dr}&\\
		A\ar{r}{\phi}&B \ar{u}\ar{r}& Z\\
		& K \ar{u}\ar[dashed]{ul}\ar{ur}&
		\end{tikzcd}
		\]
\end{proof}
\begin{lem}
	Let $\phi:A\to B$ be a morphism in an abelian categorie $\cat{A}$m and assume that $\phi$ is both a mono\morph and an epi\morph. Then $\phi$ is an iso\morph.
\end{lem}
\begin{lem}
	The inclusion $i_A:A\to A \amalg B$ is a monomorphism.
\end{lem}
\begin{proof}
	Consider the following diagram:
	\[
	\begin{tikzcd}
	Z \ar{r}{\zeta} & A\ar[bend left=30]{rr}{\id_A} \ar[swap]{r}{i_A}  & A \amalg B \ar[dashed]{r}{\exists!\pi_a} & A\\
	&&B\ar{u}{i_B} \ar[bend left = -20]{ur}{0} &
	\end{tikzcd}
	\]
	where $B$ is arbitrary and $\zeta:Z\to A$ is a morphism such that $\i_A\circ \zeta=0$. But then $\pi_A\circ i_A \circ \zeta = 0$, but $\pi_A\circ i_A = \id_A$, hence $0=\id_A\circ\zeta=\zeta$, so $i_A$ is a monomorphism.
\end{proof}
\subsection{Images; canonical decomposition of morphisms}
\begin{thm}
	Ever morphism $\pphi:A\to B$ in an abelian category $\cat{A}$ may be decomposed as
	\[
	\begin{tikzcd}
	A\ar[bend left=25]{rrr}{\pphi} \ar[\surarr]{r} & C \ar[swap]{r}{\tilde{\pphi}}&K\ar[\monarr]{r}&B
	\end{tikzcd}
	\]
	where $A\to C$ is the cokernel of the kernel of $\pphi$ and $K\to B$ is the kernel of the cokernel of $\pphi$. The induced morphism $\tilde{\pphi}:C\to K$ is uniquely determined and is an isomorphism.
\end{thm}

\begin{lem}
	Let $\pphi:A\to B$ be a morphism in an abelian category, and let $\iota:K\to B$ be the kernel of the cokernel of $\pphi$. Then
	\begin{enumerate}
		\item $\iota$ is a monomorphism
		\item $\pphi$ factors through $\iota$; and
		\item $\iota$ is initial with these property.
	\end{enumerate}
\end{lem}
\begin{proof}
	As $\iota$ is a kernel, it is automatically a monomorphism. In addition, we know that $A\to B\to \coker \pphi$ is the zero-morphism, so it factors uniquely through the kernel of the cokernel of $\pphi$:
	\[
	\begin{tikzcd}
	A\ar{r}{\pphi}\ar[dashed]{dr}\ar[bend left]{rr}{0}&B\ar{r}{\pi_B}&\coker \pphi\\
	&K\ar{u}&
	\end{tikzcd}
	\]
%	Let $\lemma:L\mmonarr B$ be any monomorphism through which $\pphi$ also factors, then $\iota$ must also factor 
\end{proof}
\begin{defn}
	Let $\pphi$ be a morphism in an abelian category. The \toidx{image} of $\pphi$ is the kernel of the cokernel of $\pphi$. The \toidx{coimage} is the cokernel of the kernel of $\pphi$. They are denoted by $\im \pphi$ and $\coim \pphi$ respectivley.
\end{defn}
\thispagestyle{rlast}


\appendix
\chapter{Übungsaufgaben}
\section{some useful stuff from one of the exercise sheets}
\backmatter
\printbibliography[heading = bibintoc]
\printindex


\end{document}