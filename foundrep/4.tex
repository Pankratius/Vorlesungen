\begin{defn}
  Let $A$ be a $k$-algebra, and $\indmodule{M}[A]$ be a right $A$-module, $\indmodule[A]{M}$ a left $A$-module.
  \begin{enumerate}
    \item Let $P$ be a $k$-module. A map
    \[
    \pphi: M\times N\to P
    \]
    is called \emph{$A$-balanced} \index{balanced map} if
    \begin{itemize}
      \item $\pphi(x+x',y)=\pphi(x,y)+\pphi(x',y)$
      \item $\pphi(x,y+y')=\pphi(x,y)+\pphi(x,y')$
      \item $\pphi(xa,y)=\pphi(x,ay)$
      \item $\pphi$ is \prf{k}{linear}.
    \end{itemize}
  \item A pair $(T,\tau)$ where $T$ is a \prf{k}{module} and $\tau$ a \prf{A}{balanced} map $M\times N\to T$ is called a \toidx{tensor product} of $M$ with $N$ over $A$, if the follwing universal property holds: For all \prf{A}{balanced} maps $\pphi:M\times N\to P$, where $P$ is any \prf{k}{module}, there is a unique \prf{k}{linear} map $f$, such that
  \[
  \begin{tikzcd}
    M\times N\ar{r}{\tau}\ar[swap]{d}{\pphi}&T\ar[dashed]{dl}{\exists!f~k\text{-linear}}\\
    P&
  \end{tikzcd}
  \]
  commutes.
\end{enumerate}
\end{defn}

\begin{lem}
  Let $A$ be a \prf{k}{algebra}, $\indmodule{M}[A]$ a right $A$-module, $\indmodule[A]{N}$ a left $A$-module.
  \begin{enumerate}
    \item There exists a tensor product $(T,\tau)$ of $M$ with $N$ over $A$.
    \item This tensor product is unique up to unique isomorphism. More precisely, if $(T',\tau')$ is any other tensor product, then there exists a unique isomorphism of \prf{k}{modules} $f:T\to T'$, such that $f\circ \tau = \tau'$.
  \end{enumerate}
\end{lem}

\begin{proof}
\begin{enumerate}
  \item \textit{existence:} Let $F$ be the free \prf{k}{module} with basis $M\times N$, and $U$ the submodule generated by elements of the form
  \begin{itemize}
    \item $(x+x',y)-(x,y)-(x',y)$
    \item $(x,y+y')-(x,y)-(x,y')$
    \item $(xa,y)-(x,ay)$
    \item $(\lambda x,y)-\lambda (x,y)$
    \item $(x,\lambda y)-\lambda(x,y)$
  \end{itemize}
  Then $F/U$ is a \prf{k}{module}, and
  \[
  \begin{tikzcd}
    \tau: M\times N\ar{r}&F\ar{r}&F/U
  \end{tikzcd}
  \]
  is \prf{A}{balanced} by definition. We set
  \[
  a\otimes b:= \tau((a,b))\text{ and }M\otimes_AN:=F/U.
  \]
  The pair $(M\otimes_A N, \otimes)$ satisfies the universal property of a tensor product:\\
  Let $\pphi:M\times \to P$ be \prf{A}{balanced}. Then there exists a unique $\hat{\pphi}$ such that the following diagram commutes, \coms as $F$ is free with basis $M\times N$\come :

  \[
  \begin{tikzcd}
    M\times N\ar[\embarr]{r}\ar[swap]{d}{\pphi}&F\ar[dashed]{dl}{\exists!\hat{\pphi}}\\
    P&
  \end{tikzcd}.
  \]
  Since $\pphi$ is \prf{A}{balanced}, $\hat{\pphi}$ factors through $F/U$, i.e.:

  \[
  \begin{tikzcd}
    M\times N\ar[\embarr]{r}\ar[swap]{d}{\pphi}&F\ar[swap]{dl}{\hat{\pphi}}\ar{r}{\pi}&F/U\ar[dashed]{dll}{\exists !\overline{\hat{\pphi}}}\\
    P&&
  \end{tikzcd}.
  \]
  Now set $\tau:=\overline{\hat{\pphi}}$.
  \item \color{red}later...\color{black}
\end{enumerate}
\end{proof}

\begin{lem}\label{4:tenspbimod}
  Let $A,B,C$ be \prf{k}{algebras}, $\indmodule[A]{M}[B]$ and $\indmodule[B]{N}[C]$ bimodules. Then $M\otimes_BN$ is a $A$-$C$-bimodule, via
  \begin{itemize}
    \item $a(x\otimes y)=(ax)\otimes y$
    \item $(x\otimes y)c=x\otimes(yc)$,
  \end{itemize}
  for all $x\in X,y\in Y,a\in A,c\in C$.
\end{lem}

\begin{proof}
  $M\otimes_BN$ is already a $k$-module. For all $a\in A$ define
  \[
  \begin{tikzcd}
    \tau_a:M\times N\ar{r}&M\times N\ar{r}{\otimes}&M\otimes_B N\\
    (x,y)\ar[mapsto]{r}&(ax,y)\ar[mapsto]{r}&(ax)\otimes y
  \end{tikzcd}
  \]
  $\tau_a$ is \prf{A}{balanced}:
  \begin{itemize}
     \item $\tau_a((x+x',y))-\tau_a((x,y))-\tau_a((x',y))= (a(x+x'))\otimes y - (ax)\otimes y - (ax')\otimes y =0$
     \item $\tau_a((xb,y))-\tau_a((x,by))=(a(xb))\otimes y -(ax)\otimes(by)=((ax)b)\otimes y - (ax)\otimes (by)= 0$
  \end{itemize}
So it factors through $M\otimes_BN$ as follows:
\[
\begin{tikzcd}
  M\times N\ar{d}{\tau_a} \ar{r}{\otimes}&M\otimes_B N\ar{dl}{\exists!f_a\text{ bilinear}}\\
  M\otimes_B N
\end{tikzcd}
\]
The map
\[
A\to \eno_k(M\otimes_BN),~a\mapsto f_a
\]
is a homomorphism of $k$-algebras. So by \cref{2:endalg}, $M\otimes_BN$ is a left \prf{A}{algebra}.\\ If we consinder the map
\[
\begin{tikzcd}
  \tau_c:M\times N\ar{r}&M\times N\ar{r}&M\otimes_BN\\
  (x,y)\ar[mapsto]{r}&(x,yc)\ar[mapsto]{r}&x\otimes(yc)
\end{tikzcd}
\]
\end{proof}
\begin{rem}
  Let $\pphi:A\to B$ be a $k$-algebra homomorphism, $\indmodule[B]{N}$ a left \prf{B}{module}. Then $M:=A$ is a $A$-$B$-bimodule via
  \[
  a.x.b:=ax\pphi(b)~\text{for all }a\in A,x\in M,b\in B.
  \]
  Then by \cref{4:tenspbimod}, $A\otimes_BN$ is a left $A$-module, where we think of $B$ as a right $k$-module. This construction is sometimes called \toidx{extension of scalars} or \toidx{induction} \emph{of $B$ by $A$}.
\end{rem}

\begin{lem}
  Let $A,B,C,D$ be \prf{k}{algebras} and consider
  \[
  \indmodule[A]{M}[B],~\indmodule[A]{(M_i)}[B]~(i\in I),~\indmodule[B]{N}[C],~\indmodule[B]{(N_j)}[C]~(j\in J).
  \]
  Then there are isomorphisms:
  \begin{enumerate}
    \item
  \[
  \begin{tikzcd}
  \left({\displaystyle\bigoplus_{i\in I}}M_i\right)\otimes_B N \ar{r}& {\displaystyle\bigoplus_{i\in I}}\left(M_i\otimes_BN\right)\\
  (x_i)\otimes y \ar[mapsto]{r}& (x_i\otimes y)\end{tikzcd}
  \]
  of $A$-$C$-bimodules.
  \item
  \[
  \begin{tikzcd}
    M\otimes_B\left({\displaystyle \bigoplus_{j\in J}}N_j\right)\ar{r}&{\displaystyle \bigoplus_{j\in J}}\left(M\otimes_BN_j\right)\\
    x\otimes (y_j)\ar[mapsto]{r}&(x\otimes y_j)
  \end{tikzcd}
  \]
  of $A$-$C$-bimodules.
  \item
  \[
  \begin{tikzcd}
  (M\otimes_BN)\otimes_CP\ar{r}&M\otimes_B(N\otimes_CP)\\
  (x\otimes y)\otimes z \ar[mapsto]{r}&x\otimes(y\otimes z)
  \end{tikzcd}
  \]
  of $A$-$B$-bimodules.
  \item
  \[
  \begin{tikzcd}
    A\otimes_AM\ar{r}&M\\
    a\otimes x\ar[mapsto]{r}&ax\\
    M\otimes_B\ar{r}&M\\
    x\otimes b\ar[mapsto]{r}&xb
  \end{tikzcd}
\] of $A$-$B$-bimodules.
\end{enumerate}
\end{lem}
\begin{proof}
  This is \color{red} supposed to be{}\color{black} exactly the same as \cite[2.27]{franzen}.
\end{proof}

\begin{prop}
  Let $A,B$ be $k$-algebras, and $M_A,\indmodule[A]{N}[B]$ and $P_b$ (bi)-modules. The map
  \[
  \hom_B(M\otimes_AN,P)\to \hom_A\left(M,\hom_B\left(N,P\right)\right),~f\mapsto \Phi(f)\text{ where }\Phi(f)(x)(y)\mapsto f(x\otimes y),
  \]
  for $x\in M$ and $y\in N$.\\
  is a well-defined isomorphism of $k$-modules, natural in $M,N,P$.
\end{prop}
\begin{proof}
  \begin{itemize}
    \item $\pphi(f)(x)$ is right $B$-module homomorphism:
    \[
      \Phi(f)(x)(yb)=f(x\otimes yb) = f((x\otimes y)b)=f(x\otimes y)b
    \]
    \item $\pphi(f)$ is right $A$-module homomorphism:
    \begin{align*}
      &\Phi(f)(xa)(y)=f(xa\otimes y) = f(a\otimes ay)\\
      &\left(\Phi(f)(x)a\right)(y) = \Phi(f)(x)(ay) =f(a\otimes ay)
    \end{align*}
    \item $\Phi$ is $k$-linear.
  \end{itemize}
  So $\Phi$ is a well-defined map of $k$-modules.\par
  $\Phi$ has inverse: Let $g\in \hom_A(M,\hom_B(N,P))$. Define
  \[
  \psi(g):M\times N\to P,~(x,y)\mapsto g(x)(y).
  \]
  Then $\psi(g)$ is $A$-balanced, so it factors through the tensor product:
  \[
  \begin{tikzcd}
    M\times N\ar{r}{\otimes}\ar[swap]{d}{\psi(g)}&M\otimes_AN\ar[swap]{dl}{\exists!\hat{\psi}(g)~k\text{-linear}}\\
    P&
  \end{tikzcd}
  \]
  The map $\hat{\pphi}(g)$ is also as right $B$-module homomorphism, \coms $g(x)\in \hom_B(N,P)$\come.\par
  But the map
  \[
  \hat{\psi}:\hom_A(M,\hom_B(N,P))\to \hom_B(M\otimes N,P)
  \]
  is the inverse of $\pphi$, as
  \begin{itemize}
    \item $\hat{\psi}(\Phi(f))(x\otimes y)=\Phi(f)(x)(y)=f(x\otimes y)$
    \item $\Phi(\hat{\psi}(g))(x)(y)=\hat{\psi}(g)(x\otimes y)=g(x)(y)$
  \end{itemize}
  for all $x\in M$ and $y\in N$.
\end{proof}

\chapter{Categories and Functors}
\begin{defn}
  A category $\cat{C}$ consists of:
  \begin{itemize}
    \item A class $\objc{C}$, whose elements are called the \toidx{objects} of $\ccat$.
    \item For all $X,Y\in \objc{C}$, a set $\ccat(X,Y)$. An element of $\ccat(X,Y)$ is called a \emph{morphism} \index{morphism!in a general category} from $X$ to $Y$ as is denoted by
    \[
    f:X\to Y.
    \]
    \item For all $X,Y,Z\in \objc{C}$, a map
    \[
    \pphi(x,y)\times \pphi(y,z)\to \pphi(x,z),~(f,g)\mapsto g\circ f.
    \]
  \end{itemize}
  These should satisfy:
  \begin{itemize}
    \item (L1): Associativity: For all $X,Y,Z,W\in \objc{C}$, and morphisms
    \[
    \begin{tikzcd}
      X\ar{r}{f}&Y\ar{r}{g}&Z\ar{r}{h}&W,
    \end{tikzcd}
    \]
    we have
    \[
    h\circ(g\circ f)=(h\circ g)\circ f .
    \]
    \item (L2): Identity: For all $X\in \objc{C}$, there is a morphism $\id_X\in \ccat(X,X)$, such that for any object $Y\in \objc{C}$:
    \[
    f\circ \id_X=f\text{ and }\id_X\circ g=g\text{ for all }f:X\to Y,g:Y\to X
    \]
    holds.
  \end{itemize}
  \end{defn}

  \begin{rem}
    \begin{enumerate}
      \item $\pphi(X,Y)=\emptyset$ can happen if $X\neq Y$
      \item $\id_X$ is unique, as $\id_X=\id_X\circ \id_{X}'=\id_X'$
    \end{enumerate}
  \end{rem}
  \begin{rem}
    We sometimes want to consider categories whose objects are all sets (with additional conditions). But this can cause logical problems. As a solution, we introduce so called universes. We will always fix a universe, such that sets are elements of this universe, and classes are subsets of this universe. Consider \cite[1.6]{catwork} for further reference.
  \end{rem}
  
