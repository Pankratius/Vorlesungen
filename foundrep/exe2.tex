\section{Sheet 2}
\begin{sol}
  Consider the two representations
  \[
  \begin{tikzcd}
    k\ar[shift left]{r}{a}\ar[shift right,swap]{r}{b}&k
  \end{tikzcd}
  ~\text{and}~
  \begin{tikzcd}
    k\ar[shift left]{r}{c}\ar[shift right,swap]{r}{d}&k
  \end{tikzcd}
  \]
  with $a,b,c,d\in k$.
Morphism of representations are in this case $k$-linear maps $k\to k$, i.e. multiplication by elements $\mu,\nu\in k$, such that the diagrams
\[
\begin{tikzcd}
  k\ar{r}{a}\ar[swap]{d}{\mu}&k\ar{d}{\nu}\\
  k\ar[swap]{r}{c}&k
\end{tikzcd}~\text{and}~
\begin{tikzcd}
  k\ar{r}{b}\ar[swap]{d}{\mu}&k\ar{d}{\nu}\\
  k\ar[swap]{r}{d}&k
\end{tikzcd}
\]
commute. This is the case if $(\mu,\nu)$ satisfies the system of equations
\[
\underbrace{\begin{pmatrix}
  c&-a\\
  d&-b
\end{pmatrix}}_{=:A}\cdot \begin{pmatrix}\mu\\\nu\end{pmatrix}=0 \Longleftrightarrow \begin{pmatrix}\mu\\\nu\end{pmatrix} \in \ker A
\]
There are several cases to consider:
\begin{itemize}
  \item $\det A=ad-bc\neq 0$: As $A$ is invertible in this case, $\ker A$ is trivial, and hence
  \[
  \hom(X_{(a,b)},X_{(c,d)})=0
  \]
  \item $\det A = 0; a=b=c=0$. As $A=0$, $\ker A = k^2$ holds, and hence
  \[
  \hom(X_{(a,b)},X_{(c,d)})=k^2\]
  \item $\det A =0;b\neq 0$ In this case, $c=ad/b$ holds.
  \begin{itemize}
    \item $a = 0,d=0 $:
    \[
    A = \begin{pmatrix}0&0\\0&-b\end{pmatrix}\implies \ker A = \lin \begin{pmatrix}1\\0\end{pmatrix}\]
    \item $a=0,d\neq 0$
    \[
    A = \begin{pmatrix}0&0\\d&-b\end{pmatrix}\implies \ker A = \lin \begin{pmatrix}b/d\\1\end{pmatrix}\]
  \end{itemize}
  \item $\det A =0, b =0$: Consider the cases:
  \begin{itemize}
    \item $c,d\neq 0,a=0$:
    \[
    A = \begin{bmatrix} c&0\\d&0\end{bmatrix}\implies \ker A = \lin \begin{pmatrix}0\\1\end{pmatrix}\]
    \item $c\neq 0, a,d=0$...
  \end{itemize}
\end{itemize}
\end{sol}

\begin{sol}
  \begin{enumerate}
    \item A representation of $V$ consits of a vector space $V$ together with an endomorphism $f\in \eno_k V$:
    \[
    X:=\begin{tikzcd}V\arrow[out=90,in=0,loop,"f"]\end{tikzcd}.
    \]
    A decomposition of $X$ into subrepresentations would correspond to a decomposition
    \[
    V = V_1\oplus V_2
    \]
    with subspaces $V_1$ and $V_2$ of $V$, such that both $V_1$ and $V_2$ are $f$-invariant.
    Now, let $X$ be any representation of $Q$.
    Assume first that $n:=\dim_kV<\infty$, and $f$ is any .
    As $k$ is algebraically closed, there is a unique (up to permutation) basis $B$ of $V$ given by a disjoint unions of Jordan-Chains
    \[
    B =\bigcup_{\lambda \in k}\bigcup_{i\in I_{\lambda}}J(\lambda,\ell_i)~\text{where}~\sum_{\lambda \in k}\sum_{\lambda \in I_{\lambda} k} \ell_i =n,\]
     $I_{\lambda}$ are finite index sets, unequal to zero for only finitely many $\lambda \in k$, and $J(\lambda,\ell_i)$ are Jordan-Chains of $f$ for the eigenvalue $\lambda$ with length $\ell_i$. This basis induces a unique direct sum decomposition
     \[
     V = \bigoplus_{\lambda \in k}\bigoplus_{i\in I_{\lambda}} \lin J(\lambda,\ell_i).
     \]
     By construction of the Jordan-Chains, $J(n,\lambda_i)$ can not be decomposed for
  \end{enumerate}
\end{sol}

\setcounter{sol}{3}
\begin{sol}
  \begin{enumerate}
    \item $R$ is a $k$-vector space,where the addition and scalar multiplication are defined component-wise. This gives $R$ the structure of a $k$-vector space, as $M$, $N$ and $X$ are in particual $k$-vector spaces.\\
    Consider now the map
    \[
    R\times R \to R,~\begin{pmatrix}a&x\\0&b\end{pmatrix}, \begin{pmatrix}a'&x'\\0&b'\end{pmatrix}\mapsto \begin{pmatrix} aa'&ax'+xb'\\0&bb'\end{pmatrix},
    \]
    where the operation $ax'+xb'$ is well-defined, as $X$ is an $A$-$B$-bimodule. This makes $R$ into a ring, as:
    \begin{itemize}
      \item the unit is given by $\begin{pmatrix}1_A&0\\0&1_B\end{pmatrix}$:
      \[
      \begin{pmatrix}1_A&0\\0&1_B\end{pmatrix} \cdot \begin{pmatrix}a&x\\0&b\end{pmatrix} = \begin{pmatrix}a&x\\0&b\end{pmatrix} = \begin{pmatrix}a&x\\0&b\end{pmatrix}\cdot \begin{pmatrix}1_A&0\\0&1_B\end{pmatrix}
      \]
      \item the multiplication is associative, as
      \begin{align*}
      \begin{pmatrix}a''&x''\\0&b''\end{pmatrix}\cdot \left( \begin{pmatrix}a'&x'\\0&b'\end{pmatrix}\cdot  \begin{pmatrix}a&x\\0&b\end{pmatrix}\right)&= \begin{pmatrix}a''&x''\\0&b''\end{pmatrix}\cdot
      \begin{pmatrix}a'a&a'x+x'b\\0&b'b\end{pmatrix} \\
      &= \begin{pmatrix} a''(a'a)\end{pmatrix}...
      \end{align*}
      \item the distributivity holds, as
    \end{itemize}
    
  \end{enumerate}

\end{sol}

\subsection*{Extension and restriction of scalars}
We do some recap from Algebra 1 (cf. \cite[27f.]{atiyah1994introduction}). For this, we go back to the commutative case: In the following, $A,B$ denote commutative, untial rings. \\
\begin{prop}
  Let $A\far{f}B$ be a ring homomorphism and $N$ a $B$-module. Then $N$ has a $A$-module structure, given by
  \[
  A\times N\mapsto N,~(a,n)\mapsto f(a)n.\]
\end{prop}
\begin{proof}
  \begin{itemize}
    \item The addition on $N_A$  is the same as the addition of $N_B$. \\
    \item Associativity: $(ab)n = f(ab)n = (f(a)f(b))n = f(a)(f(b)n)$, as $N$ is $B$-module
    \item Unit acts as unit: $1_An = f(1_A)n = 1_Bn = n$, as $f$ is homomorphism of rings.
    \item Distributivity: $(a+b)n=f(a+b)n = (f(a)+f(b))n = f(a)n+f(b)n$ and $a(n+n') = f(a)(n+n') = f(a)n+f(a)n'$
  \end{itemize}
\end{proof}
This way of obtaining a $A$-module structure on $N_B$ is called \toidx{restriction of scalars}. In particular, $f$ defines a $A$-module structure on $B$ in this way.\\
\begin{prop}
  Let $M_A$ be an $A$-module. Then
  \[
  M_B:= B\otimes_A M
  \]

  carries a $B$-module structure, and
  \[
  b(b'\otimes x) = (bb')\otimes x
  \]
  holds for this $B$-module. We say that the $B$-module $M_B$ was obtained from $M$ by \toidx{extension of scalars}
\end{prop}
