\begin{sol}
  \begin{enumerate}
    \item $\pphi_m$ is $k$-linear:
    \begin{itemize}
      \item $\pphi(a+b)x=(a+b)x\overset{(L1)}{=}ax+bx=\pphi(a)x+\pphi(b)x$
      \item $\pphi(\lambda a)=(\lambda a)x\overset{(L5)}{=}\lambda(ax)=\lambda\pphi(a)x$.
    \end{itemize}
    $\pphi_m$ is ring homomorphism:
    \begin{itemize}
      \item $\pphi(ab)=(ab)x\overset{(L3)}{=}a(bx)=\pphi(a)\pphi(b)x$
      \item $\pphi(1_A)\overset{(L4)}{=}(1_A)x=x$.
    \end{itemize}
    As these relations hold for all $x$, the assertion follows.
    \item $V_{\pphi}$ is already a $k$-module.
    \begin{itemize}
      \item (L1) $a(x+y)=(\pphi(a))(x+y)\overset{\pphi \in \eno_k(V)}{=}(\pphi(a))(x)+(\pphi(a))(y) = ax+ay$
      \item (L2) $(a+b)x=(\pphi(a+b))x\overset{\pphi\text{ homo of } k-\text{algebras}}{=}(\pphi(a)+\pphi(b))x=\pphi(a)x+\pphi(b)x=ax+bx$
      \item (L3) $(ab)x=(\pphi(ab))x\overset{\pphi\text{ homo of } k-\text{algebras}}{=}(\pphi(a)\pphi(b))x=a(bx)$
      \item (L4) $1_ax=(\pphi(1_a))x\overset{\pphi\text{ homo of } k-\text{algebras}}{=}\id x=x$
      \item (L5) $(\lambda a)(x)=((\pphi(\lambda a))x\overset{\pphi\text{ homo of } k-\text{algebras}}{=} (\lambda\pphi(a))x=\lambda(ax)$ and \\$a(\lambda x)=(\pphi(a)(\lambda x))\overset{\pphi(a)\text{ endo of } k-\text{module}}{=}\lambda (\pphi(a))x=\lambda (ax)$,
    \end{itemize}
    for all $a,b\in A,x,y\in V$ and $\lambda \in k$.
    \item We regard $V$ and $W$ as $A$-modules in the sense of part ii). Assume that $\psi(a)\circ f=f\circ \pphi(a)~(\ast)$ for all $a\in A$. Then
    \[
    f(ax)=f((\pphi(a))x)\overset{\ast}{=}(\psi(a))(f(x))=af(x),
    \]
    for all $x\in X$.
    Hence $f$ is an $A$-module homomorphism.\\
    Assume that $f$ is a a-module homomorphism, then
    \[
    (\psi(a))(f(x))=a(f(x))=f(ax)=f((\pphi(a))x)\]
    for all $x\in V$. Hence $\psi(a)\circ f - f\circ \pphi(a)=0$ and so $(\ast)$ holds.
  \end{enumerate}
\end{sol}

\begin{sol}
  \begin{enumerate}
    \item Assume that $I$ is a non-zero ideal of $A$. Let $a=(a)_{ij}\neq 0$ be an arbitrary matrix in $I$. Then there exist permutation matrices $\sigma, \pi \in \gl_n(K)$, such that $(\sigma a \pi)_{11}\neq 0$, which is in $I$, as $I$ is a two-sided ideal. So without loss of generality, suppose $a_{11}\neq 0$.\\
    Define
    \[
    b\in \mat_n(k),(b)_{ij}:= \begin{cases} 1,&\text{if }i=j=1\\0,&\text{else}\\\end{cases}
    \]
    and $E_n$ as the identity of $\mat_n(k)$. Then we get
    \[
    \left(\frac{1}{a_{11}} E_n\right)\cdot b \cdot a\cdot b = b.\]
    By repeatetly using permutation matrixes, it is possible to write any matrix as sum of products of $a$, $b$ and permutation matrices on the left- and right. As $I$ is a two-sided ideal, all of these combinations are in $I$ as well. Hence $a$ generates all of $A$, and $I=A$.
    %\item As $k$ is a field, $\mat_n(k)\cong \eno_k(k^n)$ holds for a choosen basis of $k^n$, and in particular, $\dim \eno_k(k^n)=n^2$. Hence $\dim \mat_n(k)=n^2$ as a $k$-vector space. Furthermore, $\dim \eno_k(M)=m^2$, where $m:=\dim M$ is the dimension of $M$ as a $k$-vector space.\\
    %Consider now the map
    %\[
    %\pphi:\mat_n(k)\to \eno_k(M),~a \mapsto: a:(x\mapsto ax),
    %\]
    %which maps a matrix $a$ to the linear map induced by the $\mat_n(K)$-algebra structure on $M$. This is a homomorphism of $k$-algebras, and in particular, the kernel of $\pphi$ is a two-sided ideal of $\mat_n(k)$, as
    %\[a0x=0ax=0\]
    %holds for any $a\in \mat_n(k)$.
    %Now i) implies that either $\ker \pphi=0$ or $\ker \pphi = \mat_n(k)$. But since $\pphi(E_n)=\id_{M}$, where $E_n$ is the $n\times n$ identity matrix, the latter one is not possible. Hence $\ker \pphi =0$, and $\pphi$ is an injective map of $k$-vector spaces. But this implies $\dim A\leq \dim \eno_K(V)$, so $m\geq n$.
    \item Consider $A$ as a $k$-vector space, then $\dim_KA=n^2$. Let $M$ be any left $A$-module. As shown in task 3, there is a homomorphism of $k$-algebras
    \[
    \pphi: A\to \eno_k(M),~a\mapsto a:(x\mapsto ax),
    \]
    which is in particular a homomorphism of $k$-vector spaces. The kernel of $\pphi$ is a two-sided ideal of $A$, as
    \[
    a0x=0ax=0\]
    for all $a\in A$ and $x\in M$.\\
    Now i) implies that $\ker \pphi$ is either zero or $\ker \pphi = A$. But since $\pphi(E_n)=\id_{M}$, the latter one is not possible. Hence $\pphi$ is injective, and in particular $\dim A\leq \dim \eno_k(V)$, so $n\leq m$.
  \end{enumerate}
\end{sol}
