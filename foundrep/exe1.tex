\section{Sheet 1}
\begin{sol}
  \begin{enumerate}
    \item $\pphi_m$ is $k$-linear:
    \begin{itemize}
      \item $\pphi(a+b)x=(a+b)x\overset{(L1)}{=}ax+bx=\pphi(a)x+\pphi(b)x$
      \item $\pphi(\lambda a)=(\lambda a)x\overset{(L5)}{=}\lambda(ax)=\lambda\pphi(a)x$.
    \end{itemize}
    $\pphi_m$ is ring homomorphism:
    \begin{itemize}
      \item $\pphi(ab)=(ab)x\overset{(L3)}{=}a(bx)=\pphi(a)\pphi(b)x$
      \item $\pphi(1_A)\overset{(L4)}{=}(1_A)x=x$.
    \end{itemize}
    As these relations hold for all $x$, the assertion follows.
    \item $V_{\pphi}$ is already a $k$-module.
    \begin{itemize}
      \item (L1) $a(x+y)=(\pphi(a))(x+y)\overset{\pphi \in \eno_k(V)}{=}(\pphi(a))(x)+(\pphi(a))(y) = ax+ay$
      \item (L2) $(a+b)x=(\pphi(a+b))x\overset{\pphi\text{ homo of } k-\text{algebras}}{=}(\pphi(a)+\pphi(b))x=\pphi(a)x+\pphi(b)x=ax+bx$
      \item (L3) $(ab)x=(\pphi(ab))x\overset{\pphi\text{ homo of } k-\text{algebras}}{=}(\pphi(a)\pphi(b))x=a(bx)$
      \item (L4) $1_ax=(\pphi(1_a))x\overset{\pphi\text{ homo of } k-\text{algebras}}{=}\id x=x$
      \item (L5) $(\lambda a)(x)=((\pphi(\lambda a))x\overset{\pphi\text{ homo of } k-\text{algebras}}{=} (\lambda\pphi(a))x=\lambda(ax)$ and \\$a(\lambda x)=(\pphi(a)(\lambda x))\overset{\pphi(a)\text{ endo of } k-\text{module}}{=}\lambda (\pphi(a))x=\lambda (ax)$,
    \end{itemize}
    for all $a,b\in A,x,y\in V$ and $\lambda \in k$.
    \item We regard $V$ and $W$ as $A$-modules in the sense of part ii). Assume that $\psi(a)\circ f=f\circ \pphi(a)~(\ast)$ for all $a\in A$. Then
    \[
    f(ax)=f((\pphi(a))x)\overset{\ast}{=}(\psi(a))(f(x))=af(x),
    \]
    for all $x\in X$.
    Hence $f$ is an $A$-module homomorphism.\\
    Assume that $f$ is a a-module homomorphism, then
    \[
    (\psi(a))(f(x))=a(f(x))=f(ax)=f((\pphi(a))x)\]
    for all $x\in V$. Hence $\psi(a)\circ f - f\circ \pphi(a)=0$ and so $(\ast)$ holds.
  \end{enumerate}
\end{sol}

\begin{sol}
  \begin{enumerate}
    \item Assume that $I$ is a non-zero ideal of $A$. Let $a=(a)_{ij}\neq 0$ be an arbitrary matrix in $I$. Then there exist permutation $\sigma, \pi \in \gl_n(K)$ matrices, such that $(\sigma a \pi)_{11}\neq 0$, which is in $a$, as $I$ is a two-sided ideal. So without loss of generality, suppose $a_{11}\neq 0$.\\
    Define
    \[
    b\in \mat_n(k),(b)_{ij}:= \begin{cases} 1,&\text{if }i=j=1\\0,&\text{else}\\\end{cases}
    \]
    and $E_n$ as the identity of $\mat_n(k)$. Then we get
    \[
    \left(\frac{1}{a_{11}} E_n\right)\cdot b \cdot a\cdot b = b.\]
    By repeatetly using permutation matrixes, it is possible to write any matrix as sum of products of $a$, $b$ and permutation matrices on the left- and right. As $I$ is a two-sided ideal, $a$ all of these combinations are in $I$ as well. Hence $a$ generates all of $A$, and $I=A$.
    %\item As $k$ is a field, $\mat_n(k)\cong \eno_k(k^n)$ holds for a choosen basis of $k^n$, and in particular, $\dim \eno_k(k^n)=n^2$. Hence $\dim \mat_n(k)=n^2$ as a $k$-vector space. Furthermore, $\dim \eno_k(M)=m^2$, where $m:=\dim M$ is the dimension of $M$ as a $k$-vector space.\\
    %Consider now the map
    %\[
    %\pphi:\mat_n(k)\to \eno_k(M),~a \mapsto: a:(x\mapsto ax),
    %\]
    %which maps a matrix $a$ to the linear map induced by the $\mat_n(K)$-algebra structure on $M$. This is a homomorphism of $k$-algebras, and in particular, the kernel of $\pphi$ is a two-sided ideal of $\mat_n(k)$, as
    %\[a0x=0ax=0\]
    %holds for any $a\in \mat_n(k)$.
    %Now i) implies that either $\ker \pphi=0$ or $\ker \pphi = \mat_n(k)$. But since $\pphi(E_n)=\id_{M}$, where $E_n$ is the $n\times n$ identity matrix, the latter one is not possible. Hence $\ker \pphi =0$, and $\pphi$ is an injective map of $k$-vector spaces. But this implies $\dim A\leq \dim \eno_K(V)$, so $m\geq n$.
    \item Consider $A$ as a $k$-vector space, then $\dim_KA=n^2$. Let $M$ be any left $A$-module. As shown in task 3, there is a homomorphism of $k$-algebras
    \[
    \pphi A\to \eno_k(M),~a\mapsto a:(x\mapsto ax),
    \]
    which is in particular a homomorphism of $k$-vector spaces. The kernel of $\pphi$ is a two-sided ideal of $A$, as
    \[
    a0x=0ax=0\]
    for all $a\in A$ and $x\in M$.\\
    Now $i)$ implies that $\ker \pphi$ is either zero or $\ker \pphi = A$. But since $\pphi(E_n)=\id_{M}$, the latter one is not possible. Hence $\pphi$ is injective, and in particular $\dim A\leq \dim \eno_k(V)$, so $n\leq m$.
  \end{enumerate}
\end{sol}

\hrule
\begin{prop}
Let $k$ be a field, $k[X]$ the polynomial ring and $p\in k[X]$ a polynomial with $\deg p=n$. Then
	\[
	k[X]/(p)
	\]
is a $n$-dimensional $k$ vector space, and a basis is given by
	\[
	\{1,x,...,x^{n-1}\}.
	\]
\end{prop}
The following propositions are taken from \cite{aluffi}.\\
Let $R$ be any commutative ring.
\begin{prop}
  Let $I_1,...,I_k$ be ideals of $R$ such that $I_i+I_j=(1)$ for all $i\neq j$. Then the natural homomorphism
  \[
  \pphi :R\to R/I_1\times \hdots \times R/I_k
  \]
  is surjective and induces an isomorphism
  \[
  \frac{R}{I_1\hdots I_k}\to R/I_1\times \hdots \times R/I_k
  \]
\end{prop}
\begin{cor}[Chinese remainder theorem]
  Let $R$ be a PID and $a_1,...,a_k\in R$ be elements such that $\gcd(a_i,a_j)=1$ for all $i\neq j$. Let $a=a_1\hdots a_k$. Then the function
  \[
  \pphi: R/(a)\to R/(a_1)\times \hdots \times R/(a_k).
  \]
\end{cor}
\begin{prop}[Yoneda Lemma\index{Lemma!Yoneda}]
	Let $\cat{C}$ be a category, $X$ an object of $\cat{C}$ and consider the contravariant functor
	\[
	h_X:= \hom_{\cat{C}}(-,X).
	\]
	Then for every contravariant functor $\fcr{F}:\cat{C}\to \cat{Set}$, there is a bijection between the set of natural transformations $h_x\nattraf\fcr{F}$ and $\fcr(X)$.
\end{prop}
\begin{defn}[\cite{assem1}]
	The (Jacobson) \emph{radical} $\rad A$ of a $K$-algebra $A$ is the intersection of all maximal right ideals in $A$. It is the same as the intersetion of all left-sided maximal right ideals in $A$. Furthermore, $\rad A$ is a two-sided ideal.
\end{defn}

\begin{defn}
	Let $f,g:X\to Y$ be morphisms in a category $\cat{C}$. Then a morphism $e:E\to X$ is called \emph{equalizer} of $f$ and $g$ if $f\circ e=g\circ e$ and for all other morphisms $o:O\to X$, such that $f\circ o=g\circ o$, there is a unique morphis $O\to E$, such that the following diagram commutes:
	\[
	\begin{tikzcd}
		E\ar{r}{e}&X\ar[shift left = .5ex]{r}\ar[shift right = .5ex]{r}&X\\
		O\ar[dashed]{u}{\exists !}\ar{ur}{o}&&.
	\end{tikzcd}
	\]
\end{defn}
\begin{prop}
	Equalizers exists in abelian categories.
\end{prop}
\hrule
\subsection*{Task 1}
See this as a functors:
\[
\begin{tikzcd}k\mdas\alc \ar[shift left]{r}{\fcr{V}}\ar[shift right, swap,leftarrow]{r}{\fcr{A}}&\grpc\end{tikzcd}
\]
Grp. alg. construction $\fcr{A}$ is left-adjoint to group of units construction $\fcr{V}$.
\begin{enumerate}
	\item Show that there is a natural isomorphism
	\[
	k\mdas\alc(\fcr{A}(G),A) \cong \grpc(G,\fcr{V}(A))
	\]
	for all groups $G$ and $k$-algebras $A$.
\end{enumerate}
\subsection*{Task 2}
This quiver is called the \emph{linear oriented quiver}\index{quiver!linear oriented}.
Define
\[
\pphi:KQ\to\lowtri_n(k)
\]
as linear extension of the $k$-linear map
\[
Q\bast\to \lowtri_n(k),~p_{ij}\mapsto (E_{ij}){kl}:= \begin{cases}1&\text{if }k=i,l=j\\0&\text{otherwise}\end{cases}
\]
This is indeed a homomorphism of $k$-algebras, which sends basis vectors to basis vectors.\par
\subsection*{Task 3}
Let $\ccat,\dcat$ be two \emph{$k$-linear} categories \index{category!linear}, i.e. $\ccat(X,Y)$ has the structure of a $k$-vector space and composition is bilinear. We say that $\ccat$ is \emph{equivalent} to $\dcat$ ($\ccat \simeq \dcat$) if there are \emph{$k$-linear} functors
\[
\ffunc :\ccat \to \dcat \text{ and }\gfunc:\dcat \to \ccat
\]
(i.e. functors that induce $k$-linear maps $\ccat(X,Y)\to \dcat(\ffunc(X),\gfunc(Y))$, and for $\gfunc$ analogous), such that there are natural isomorphisms
\[
\gfunc \ffunc \simeq \idfunc_{\ccat}\text{ and }\ffunc \gfunc \simeq \idfunc_{\dcat}
\]
\begin{thm}
	Let $\ccat,\dcat$ be $k$-linear categories. $\ccat \simeq \dcat$ if and only if there is a fully faithful, $k$-linear and dense functor
	\[
	\ffunc: \ccat \to \dcat .
	\]
\end{thm}
\begin{rem}
	The proof is supposed to only invoke the Axiom of Choice, and should work for general $\hom$-Sets.
\end{rem}
This task basically shows that there is an equivalence of categories
\[
\repc(A)\simeq k\mdas \mdc(A),
\]
where
\[
\repc(A):= \{(V,\pphi)\ssp V\text{ a }k\mdas\text{vector-space, }\pphi:A\to \eno_{k}(V)\text{ a algebra-homomorphism}\}
\]
Calvin highly recommends the book \cite{assem1}.
