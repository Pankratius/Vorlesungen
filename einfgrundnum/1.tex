\chapter{Orthogonalisierungsverfahren}


Betrachte $A\in \glr{n}$, wobei $A$ schlecht konditioniert sein kann. Wir wollen ein Gleichungssystem der Form $Ax=b$, mit $b\in \rr^n$ gegeben, lösen. Dazu suchen wir eine Orthogonalmatrix $Q\in \orr{n}$ und eine obere Dreiecksmatrix $R\in \mnr{n}$ mit $A=QR$. Diese Zerlegung von $A$ nennt man \toidx{Orthogonalzerlegung}. Dann erhalten wir das äquivalente Problem
\[
Ax=b \Longleftrightarrow QRx=b \Longleftrightarrow Rx=Q^T b.
\]


\section{Eigenschaften orthogonaler Matrizen}
\begin{lem}\label{1:iso}
	Sei $Q\in \orr{m}$ orthogonal. Dann ist auch $Q^T$ orthogonal und es gilt
	\[
		\eukno{Qx}= \eukno{Q^Tx}=\eukno{x}
	\]
\end{lem}
\begin{proof}
	Es gilt
	\[
	\eukno{Qx}^2 = x^TQ^TQx = x^Tx=\eukno{x}.
	\]
	Genauso für $Q^T$.
\end{proof}
\begin{lem}
	Sei $A\in \glr{n}$ regulär und $Q\in \orr{n}$ orthogonal. Dann gilt
	\[ \kappa_2(QA)=\kappa_2(A)
	\]
\end{lem}
\begin{proof}
	Die Matrixnorm $\eukno{A}$ ist durch die euklidsche Norm induziert, i.e.
	\[
	\eukno{A} = \max_{x\neq 0} \frac{\eukno{Ax}}{\eukno{x}}.
	\]
	Also folgt aus \cref{1:iso}, dass $\eukno(QA)=\eukno(A)$ gilt.\\
	Betrachte jetzt
	\begin{align*}
		\eukno{A^{-1}Q^T} &= \max_{x\neq 0}\frac{\eukno{A^{-1}Q^Tx}}{\eukno{x}}
										&= \max_{x\neq 0}\frac{\eukno{A^{-1}Q^Tx}}{\eukno{Q^Tx}}
										&\overset{y:=Q^Tx}=  \max_{y\neq 0} \frac{\eukno{A^{-1}}}{\eukno{y}} = \eukno{A^{-1}}
	\end{align*}
\end{proof}
Also ist für das LGS $Rx=Q^Tb:$ $\kappa_2(R)=\kappa_2(A)$. Also hat sich die Kondition des Problems nicht verschlechtert.

\section{Anwendung: Lineare Ausgleichsgeraden}

Betrachte für gegebenes $b\in \rr^n$ und $A\in \mnr{n}$ das Optimierungsproblem
\begin{equation}\label{1:opt}\tag{O}
\min_{x\in \rr^n} \eukno{Ax-b}.
\end{equation}
Dieses Problem ist äquivalent zur Optimierung von $\eukno{Ax-b}^2$. \par
Seien nun $m$ Tupel $(y_i,f_i)\in \rr^2$ ($1\leq i\leq m$) gegeben. Gesucht ist diejenige affine Gerade $c+dy$ in $\rr^2$, so dass die Summe der Quadrate der Punkte von der Gerade minimal ist. Wir erhalten also das Optimierungsproblem
\[
\min_{(c,d)\in \rr^2} \left(\sum_{i=1}^m(c+dy_i-f_i)^2\right) = \min_{(c,d)\in \rr^2} \eukno{\begin{pmatrix}1 &y_1\\\vdots &\vdots\\1&y_m\end{pmatrix}\cdot \begin{pmatrix}c\\d\end{pmatrix}-\begin{pmatrix}f_1\\\vdots\\f_m\end{pmatrix}}.
\]
Betrachte allgemeiner das Polynom
\[
p(y) = \sum_{k=0}^{n-1}a_ky^k.
\]
Gesucht sind jetzt die Koeffizienten $a_0,...,a_{n-1}$ mit
\[
\sum_{j=1}^m\left(p(y_j)-f_j\right)^2
\]
ist minimal. Schreibe dies ebenfalls als Optimierungsproblem:
\[
\min_{a_0,...,a_{n-1}} \eukno{\begin{pmatrix}y_1^0&\hdots&y_1^{n-1}\\\vdots&\ddots&\vdots\\y_m^0&\hdots& y_m\end{pmatrix}\cdot \begin{pmatrix}a_0\\\vdots\\a_{n-1}\end{pmatrix}-\begin{pmatrix}f_1\\\vdots\\f_m\end{pmatrix}}^2.
\]
\lec
