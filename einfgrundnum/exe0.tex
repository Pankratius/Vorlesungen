\begin{defn}
	Die \emph{Frechet-Ableitung} bezeichnet die gewöhnliche totale Ableitung einer Funktion $\rr^n\to \rr$.
\end{defn}
\begin{defn}
Sei $A\in \mat_n(\mathbb{K})$. Der \toidx{Spektralradius} von $A$ ist definiert als
\[
\rho(A):= \max \{|\lambda_1|,...,|\lambda_n|\},
\]
wobei $\lambda_1,...,\lambda_n$ die (möglicherweise komplexen) Eigenwerte von $A$ darstellen.
\end{defn}
\begin{prop}\label{exe0:specr}
  Sei $M\in \mat_n(\rr)$, und
  \[
    \pphi:\rr^n\to \rr^n,x\mapsto Mx+c.
  \]
  Dann sind äquivalent:
  \begin{enumerate}
    \item Für den Spektralradius $p$ gilt $p(M)<1.$
    \item Die Fixpunktiteration
    \[
    x_{k+1}:= \pphi(x_k)\]
    konvergiert für ein beliebiges $x_0\in \rr^n$.
  \end{enumerate}
\end{prop}
\setcounter{sol}{1}
\begin{sol}
  Es gilt für das in der Aufgabenstellung spezifizierte $\psi$:
  \[
  \{\text{Eigenwerte von }\psi\} = \lambda + (1-\lambda)\cdot\{\text{Eigenwerte von }\pphi\},
  \]wobei jeweils nur der lineare Teil betrachtet wurde. Nutze nun \cref{exe0:specr}.
\end{sol}
\begin{prop}
  Das Jacobi-Verfahren für die Matrix
  \[
  A=D-L-R
  \]
  konvergiert, falls für die Matrix
  \[
  I_{\text{Jac}.}:= D^{-1}(L+R)
  \]
  der Spektralradius größer 1 ist. Es konvergiert nicht, wenn der Spektralradius kleiner 1 ist.
\end{prop}
\begin{defn}
  Die \emph{Newton-Iteration} ist gegeben durch
  \[
  x_{k+1}:= x_k - \frac{f(x_k)}{f'(x_k)}
  \]
\end{defn}
