\documentclass[11pt,twoside]{memoir}


\usepackage{packages/setup}
\usepackage{packages/styles}

\usepackage{packages/theorems}


\title{Dokumentation}
\pretitle{\pagestyle{empty}\begin{center}  \Huge \bfseries}
	\renewcommand{\maketitlehookb}{\centering \Large  \textit{Vorlesungsmitschriften im Pankratius-Framework}}
\date{}



\begin{document}
	
\frontmatter
\maketitle
\tableofcontents
\chapter*{Vorwort}
In dieser kurzen Dokumentation sollen wichtige Informationen über das Erstellen von Vorlesungsnotizen in diesem Framework gesammelt werden.

\mainmatter
\chapter{Struktur}
\section{Dokumentenstruktur}
\begin{itemize}
\item Das Dokument ist prinzipiell in 3 Teile gegliedert - \verb|frontmatter|, \verb|mainmatter|, \verb|backmatter|. In \verb|frontmatter| sollte momentan nun die Titelseite, sowie Inhaltsverzeichnis und Vorwort. In \verb|mainmatter| befinden sich die eigentlichen Notizen, und in \verb|backmatter| momentan ein Index.
\item Neue Kapitel kann man mit dem \verb|\chapter|-Command erstellen. Sie beginnen immer auf einer neuen Seite.
\item Neue Sections kann man mit dem \verb|\section|-Command erstellen. Sie werden immer zentriert dargestellt.
\item Neue Subsections kann mit dem \verb|subsection|-Command erstellen. Momentan erhalten sie eine Nummer, werden aber nicht im Inhaltsverzeichnis dargestellt. Dies kann man über Anpassen von \verb|\setsecnumdepth{subsection}| bzw. Hinzufügen von \verb|\maxtocdepth{subsection}| ändern.
\item in \verb|styles.sty| ist das Aussehen der einzelnen Überschriften geregelt, jeweils unter \verb|\set<sec/subsec/subsubsec><feature>|.
\end{itemize}



\backmatter
\printindex

\end{document}