\chapter{Nullstellen von Polynomen}
\section{Reelle und komplexe Wurzeln}
Sei $n\in \nne$. Für $z\in \cc$ heißen die Nullstellen von $X^n-z\in \cc[X]$ die $n$-ten \toidx{Wurzeln} von $z$.
Sei $r\in \rr,r\geq 0$. Dann existiert eine eindeutig bestimmte reelle Zahl $w\geq 0$ mit $w^n=r$. Wir schreiben $\sqrt[n]{r}:=w$.\par
Sei $z=r\exp(i\alpha)$ mit $r\geq 0$ reell und $\alpha \in [0,2\pi)$ eine komplexe Zahl. Setze
\[
\sqrt[n]{z}:= \sqrt[n]{r}\exp(i\alpha/n),
\]
und definiere $\sqrt[2]{z}=:\sqrt{z}$. Definiere weiterhin
\[
\zeta_n:= \exp(2\pi i/n).
\]
Die Menge der $\zeta_n^j$ für $0\leq j\leq n-1$ heißen $n$-te \toidx{Einheitswurzeln}. Dies sind gerade die Nullstellen von $X^n-1\in \cc[X]$.

\section{Formeln für Nullstellen vom Grad $n\leq 4$}
c.f. Formeln von Cardano und Ferrari.

\section{Körpererweiterungen}
\begin{defn}
	Sei $L$ ein Körper. Eine Teilmenge $K\subseteq L$ heißt \toidx{Teilkörper} von $L$, falls gilt
	\begin{enumerate}
		\item $0,1\in K$
		\item $a+b\in K,~ab\in K$ für alle $a,b\in K$
		\item $-a\in K$ für alle $a\in K$
		\item $a^{-1}\in K$ für alle $a\in K^{\times}$
	\end{enumerate}
\end{defn}
Durch Einschränkung der Addition und Multiplikation ist $K$ wieder ein Körper. In diesem Fall heißt das Tupel $L/K:=(K,L)$ eine \toidx{Körpererweiterung}. Wir dafür auch $K\subset L$ und sagen, dass $L$ eine Körpererweiterung von $K$ ist.
\begin{lem}
	Sei $L$ ein Körper, $I$ eine Indexmenge und $K_i$ ein Teilkörper für alle $i\in I$. Dann ist
	\[
	K:= \bigcap_{i\in I} K_i
	\]
	wieder ein Teilkörper.
\end{lem}
\begin{proof}
	Wir rechnen schnell die Eigenschaften nach:
	\begin{enumerate}
		\item Es gilt $0,1\in K_i$ für alle $i$, also auch $0,1\in K$.
		\item Seien $a,b\in K_i$ für alle $i$. Dann ist auch $a+b$ und $ab$ in $K_i$ für alle $i$, weil die $K_i$ jeweils Teilkörper sind. Also auch $a+b,ab\in K$.
		\item genauso.
		\item genauso.
	\end{enumerate}
\end{proof}
\begin{defn}
	Sei $L$ ein Körper, $M\subseteq L$ eine Teilmenge. Dann ist
	\[
	(M):= \bigcap_{M\subseteq F\subseteq L} F,
	\]
	mit $F$ Teilkörper, der von $M$ \emph{erzeugte Teilkörper} \index{Teilkörper!erzeugter}. \\
	Sei $L/K$ eine Körpererweiterung und $M\subseteq L$ eine Teilmenge von $L$. Definiere
	\[
	K(M):= (K\cup M)\subset L.
	\]
	$K(M)$ entsteht an $K$ durch \toidx{Adjungtion} von $M$.
\end{defn}
\begin{bsp}
	Betrachte $\rr/\qq$ und
	\[
	\qq(\sqrt{2}) := \{ a+b\sqrt{2}\ssp a,b\in \qq\}.
	\]
\end{bsp}
Für $M = {x_1,...,x_n}\subset L$ endlich schreibt man vereinfacht auch
\[
K(M)=: K(x_1,...,x_n).
\]
\begin{defn}
	$L/K$ ist \emph{einfach} \index{Körpererweiterung!einfache}, falls es ein $x\in L$ gibt mit $K(x)=L$.
\end{defn}
\begin{defn}
	Für $M=\{0,1\}$ heißt $P:= (M)$ der \toidx{Primkörper} von $L$. Jeder Teilkörper von $L$ enthält $p$
\end{defn}
\begin{bsp}
	Der Primkörper von $\mathbb{C}$ ist $\mathbb{Q}$.
\end{bsp}
\section{Auflösbarkeit durch Radikale}
\begin{defn}
	$L/K$ ist eine \toidx{Radikalerweiterung} von $L$ falls gilt:
	\begin{enumerate}
		\item es gibt $x_1,...,x_n\in L$ mit $K(x_1,...,x_n) = L$
		\item Es gibt $r_1,...,r_n\in \nne$ mit $x_1^{r_1}\in K$ und $x_i^{r_i}\in K(x_1,...,x_{i-1})$ für $2\leq i\leq n$.
	\end{enumerate}
	Wir sagen: \glqq $L$ entsteht aus $K$ durch sukzessive Adjungtion von Wurzeln\grqq.
\end{defn}
\begin{defn}
	$f\in K[X]$ ist \emph{durch Radikale auflösbar}, falls es eine Radikalerweiterung $L/K$ gibt, so dass $f$ eine Nullstelle in $L$ hat.
\end{defn}
\begin{thm}[Cadano und Ferrari]
	Sei $f\in \qq[X]$ mit $\deg f\leq 4$. Dann ist $f$ auflösbar.
\end{thm}
Ein Ziel der Vorlesung wird es sein, zu verstehen, wann $f\in \qq[X]$ durch Radikale auflösbar ist.

\chapter{Konstruktion mit Zirkel und Lineal}
Ein weiteres Ziel der Vorlesung wird es sein, die klassischen Konstruktionsprobleme zu untersuchen.\par
Identifiziere $\cc \cong \rr^2$. Gegeben sei nun $M\subset \cc$ mit $|M|\geq 2$. Definiere
\begin{enumerate}
	\item $G(M) := \{\text{affine Geraden }G\text{ in }\rr^2\text{ mit }|G\cap M|\geq 2\}$
	\item $C(M) := \{\text{reelle Kreise }C\text{ in }\rr^2\text{mit: Mittlepunkt von }C\in M\text{ und Radius von }C=||z_1-z_2||,z_1,z_2\in M\}$
\end{enumerate}
Durch folgende Operation erhalten wir wieder komplexe Zahlen:
\begin{itemize}
	\item $(ZL1)$: Schnitt zweier Geraden aus $G(M)$
	\item $(ZL2)$: Schnitt einer Gerade aus $G(M)$ und eines Kreises aus $C(M)$
	\item $(ZL3)$: Schnitt zweier Kreise aus $C(M)$.
\end{itemize}
Setze
\[
ZL(M):= \{z\in \cc\ssp z\in M\text{ oder }z\text{ entsteht aus }M\text{durch Anwendung von }(ZL1,(ZL2),(ZL3)\}.
\]
Definiere jetzt
\[
M_0 := M, M_{i+1}:= ZL(M_i)\text{ und }M_{\infty}:= \bigcup_{i\geq 0} M_i.
\]
Die Elemente von $M_{\infty}$ heißen die von $M$ durch Zirkel und Lineal \toidx{konstruierbare Punkte}.
\begin{lem}
	$M_{\infty} = ZL(M_{\infty})$.
\end{lem}
