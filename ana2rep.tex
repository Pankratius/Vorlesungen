\documentclass[11pt,twoside]{memoir}


\usepackage{packages/setup}
\usepackage{packages/bibconfig}
%loads bibliography package for this document
\addbibresource{einfalg/lit.bib}
\usepackage{packages/styles}

\usepackage{packages/theorems}
\usepackage{packages/mymathtools}


\title{Analysis II-Revelations}
\pretitle{\pagestyle{empty}\begin{center}  \Huge \bfseries}
	\renewcommand{\maketitlehookb}{\centering \Large  \textit{Sommersemester 2018}}
\date{}

\makeatletter
\newlength{\@thlabel@width}%
\newcommand{\thmenumhspace}{\settowidth{\@thlabel@width}{\itshape1.}\sbox{\@labels}{\unhbox\@labels\hspace{\dimexpr-\leftmargin+\labelsep+\@thlabel@width-\itemindent}}}
\makeatother

\begin{document}
	
\frontmatter
\maketitle
\begin{KeepFromToc}
	\tableofcontents
\end{KeepFromToc}

\mainmatter
\chapter{Metrische Räume}
\section{Vollständigkeit}
\begin{prop}
	Sei $(X,d)$ ein kompakter metrischer Raum. Dann ist $(X,d)$ vollständig.
\end{prop}
\chapter{Differenzierbarkeit}
\section{Differenzierbarkeit allgemein}
\begin{bsp}
	Die Funktion 
	\[ f:\mathbb{R}^2\to \mathbb{R},~x\mapsto \begin{cases} \frac{x^2y^2}{x^2+y^2}&(x,y)\neq(0,0)\\0&(x,y)=0\\\end{cases}\]
	ist in $\mathbb{R}^2$ total differenzierbar.
\end{bsp}
\begin{proof}
	Es gilt bspw.
	\[\frac{\partial f}{\partial x} = \begin{cases}\frac{xy^4}{(x^2+y^2)^2}&(x,y)\neq 0\\0&(x,y)=0\\\end{cases}.\]
	Es reicht jetzt, 
	\[
	\lim_{(x,y)\to 0} \frac{\partial f}{\partial x} =0
	\]
	für die Fallunterscheidungen $x\geq y$ und $x<y$ zu zeigen, weil damit bereits alle möglichen Teilfolgen betrachtet wurden.
\end{proof}

\chapter{Mannigfaltigkeiten}
\section{Manngifaltigkeiten allgemein}
\begin{prop}
	Es gibt keinen Homöomorphismus $\mathbb{R}^m\to \mathbb{R}^n$, falls $m\neq n$.
\end{prop}
\begin{cor}
	Die Dimension einer zusammenhängenden Untermannigfaltigkeit des $\mathbb{R}^d$ ist eindeutig.
\end{cor}
\appendix
\backmatter
\printbibliography[heading = bibintoc]
\printindex


\end{document}