\documentclass[11pt,twoside]{memoir}


\usepackage{packages/setup}
\usepackage{packages/styles}

\usepackage{packages/theorems}


\title{Introduction to Algebra}
\pretitle{\pagestyle{empty}\begin{center}  \Huge \bfseries}
	\renewcommand{\maketitlehookb}{\centering \Large  \textit{Lecture Notes in the Winter Semester 2018/19}}
\date{}




\begin{document}
	
\frontmatter
\maketitle
\tableofcontents
\chapter*{Vorwort}
Diese Vorlesungsmitschriften werden in der Vorlesung \textit{Einführung in die Algebra} von Prof. Jan Schröer im Wintersemester 2018/19 an der Universität Bonn angefertigt.\par
Wir versuchen, diese immer unter \url{https://pankratius.github.io/Vorlesungen/} bei GitHub zu aktualisieren. Für Verbesserungsvorschläge sind wir über das \textit{Issue}-feature dankbar.
\mainmatter
\chapter{Gruppen}

\section{Definitionen - Grundlegendes}
\lipsum[1-5]
\section{Abbildungen von Gruppen}
\subsec{Gruppenhomomorphismen}
\lipsum[1]
\subsec{Gruppenisomorphismen}
\begin{lem}\label{thm:Apfel}
	Ich bin ein Apfel.
\end{lem}
\lec

\begin{defn}
	Ein \toidx{Gruppen\isom} ist ein Isomorphismus in der Kategorie der Gruppen. \Cref{thm:Apfel}
\end{defn}
\newpage
\section{Normalteiler und Quotientengruppen}
\begin{prop}
	Normalteiler sind ziemlich cool.
\end{prop}
\lec
\backmatter
\printindex

\end{document}