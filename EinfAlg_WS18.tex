\documentclass[11pt,twoside]{memoir}


\usepackage{packages/setup}
\usepackage{packages/bibconfig}
%loads bibliography package for this document
\addbibresource{einfalg/lit.bib}
\usepackage{packages/styles}

\usepackage{packages/theorems}
\usepackage{packages/mymathtools}


\title{Introduction to Algebra}
\pretitle{\pagestyle{empty}\begin{center}  \Huge \bfseries}
	\renewcommand{\maketitlehookb}{\centering \Large  \textit{Lecture Notes in the Winter Semester 2018/19}}
\date{}


\makeatletter
\newlength{\@thlabel@width}%
\newcommand{\thmenumhspace}{\settowidth{\@thlabel@width}{\itshape1.}\sbox{\@labels}{\unhbox\@labels\hspace{\dimexpr-\leftmargin+\labelsep+\@thlabel@width-\itemindent}}}
\makeatother

\begin{document}
	
\frontmatter
\maketitle
\begin{KeepFromToc}
	\tableofcontents
\end{KeepFromToc}
\chapter*{Vorwort}
Diese Vorlesungsmitschriften werden in der Vorlesung \textit{Einführung in die Algebra} von Prof. Jan Schröer im Wintersemester 2018/19 an der Universität Bonn angefertigt.\par
Wir versuchen, diese immer unter \url{https://pankratius.github.io/Vorlesungen/} bei GitHub zu aktualisieren. Für Verbesserungsvorschläge sind wir über das \textit{Issue}-feature dankbar.
\mainmatter
\chapter{Gruppen}

\section{Definitionen - Grundlegendes}
\lipsum[1-5]
\section{Abbildungen von Gruppen}
\subsec{Gruppenhomomorphismen}
\lipsum[1]
\subsec{Gruppenisomorphismen}
\begin{lem}\label{thm:Apfel}
	Ich bin ein Apfel.
\end{lem}
\lec

\begin{defn}
	Ein \toidx{Gruppen\isom} ist ein Isomorphismus in der Kategorie der Gruppen. \Cref{thm:Apfel}
\end{defn}
\newpage
\section{Normalteiler und Quotientengruppen}
\begin{prop}
	Normalteiler sind ziemlich cool.
	\[
	\begin{tikzcd}
	G \ar{r}{\phi} \ar[swap]{d}{\pi} & H\\
	G/N \ar[swap]{ur}{\exists! f}
	\end{tikzcd}
	\]
	Wir erhalten für den Fall, dass $\phi$ surjektiv ist, folgendes Bild:

	\[
	\begin{tikzcd}
	G \ar[\surarr]{r}{\pi}  \ar[bend left = 25]{rrr}{\phi}& G/\ker\phi \ar{r}{\cong} & \im \phi \ar[\embarr]{r}{} & H
	\end{tikzcd}
	\]
\end{prop}
\lec
\begin{defn}
	Ein kommutatives Diagramm
	\[
	\begin{tikzcd}
	0\ar{r}&M_1\ar{r}&N\ar{r}&M_2\ar{r}&0
	\end{tikzcd}
	\]
	heißt \toidx{split}, falls es isomorph zu einem Diagramm der Form
	\[
	\begin{tikzcd}
	0\ar{r}&M_1\ar{r}\ar{d}{\sim}&N\ar{r}\ar{d}{\sim}&M_2\ar{r}\ar{d}{\sim}&0\\
	0\ar{r}&M_1'\ar{r}&M_1'\oplus M_2'\ar{r}&M_2'\ar{r}&0
	\end{tikzcd}
	\]
	ist.
\end{defn}
\begin{prop}
	Sei $f\colon M\to N$ ein Homomorphismus zwischen $A$-Moduln $M,N$. Dann hat $f$ ein linksinverses, genau dann wenn das Diagramm
	\[
	\begin{tikzcd}
	0\ar{r}& M \ar{r}{f}& N\ar{r}&\mathrm{Coker}(f)\ar{r}&0
	\end{tikzcd}
	\]
	split ist.
\end{prop}
\begin{proof}
	Angenommen, $f$ hat ein linksinverses, so dass 
	\[
	\begin{tikzcd}
	0\ar{r}&M\ar{r}{f}\ar[swap,equal]{dr}&N\ar{d}{\psi}\\
			&&M
	\end{tikzcd}
	\]
	kommutiert.
\end{proof}
\begin{thm}[Zerlegungssatz für regülare Metriken]
	Sei $(V,s)$ ein regulärer symmetrischer $K$-Vektorraum, und $U\subset V$ ein Untervektorraum. Dann gilt
	\[\dim(V) = \dim(U) + \dim(U^{\perp})\]
	Ist $s_u$ regulär, so gilt außerdem 
	\begin{enumerate}
		\item $V = V \perp U^{\perp}$
		\item $s_{U^{\perp}}$ ist regulär
	\end{enumerate}
\end{thm}
Wir betrachten nun Isometrien eines metrischen Vektorraums. Dafür definieren wir die \toidx{Isometriegruppe} eines metrischen Vektorraumes $(V,s)$ über dem Körper $K$ mit Involution $\overline{\cdot}$ als
\[
\gl(V,s):= \{f:V\to V\ssp f~\text{ist eine Isometrie}\}
\]
Für isometrische Vektorräume gelten folgenden äquivalente Bedingungen
\begin{prop}
	Seien $(V,s_V)$ und $(W,s_W)$ endlich-dimensionale metrische $K$-Vektorräume bezüglich der Involution $\overline{\cdot}$. Dann sind äquivalent:
	\begin{enumerate}
		\item $(V,s_V)\cong (W,s_W)$.
		\item Es gibt Basen $B$ bzw. $C$ von $V$ bzw. $W$, so dass $\mathbf{c}_B(s_V) = \mathbf{c}_C(s_W)$.
		\item Für alle Basen $B$ bzw. $C$ von $V$ bzw. $W$, ist $\mathbf{c}_B(s_V) \equiv \mathbf{c}_C(s_W)$.
	\end{enumerate}
\end{prop}
\thispagestyle{rlast}


\appendix
\chapter{Übungsaufgaben}
\section{some useful stuff from one of the exercise sheets}
\backmatter
\printbibliography[heading = bibintoc]
\printindex


\end{document}