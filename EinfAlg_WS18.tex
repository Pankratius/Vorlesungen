\documentclass[11pt,twoside]{memoir}


\usepackage{packages/setup}
\usepackage{packages/styles}

\usepackage{packages/theorems}
\usepackage{packages/mymathtools}


\title{Einführung in die Algebra}
\pretitle{\pagestyle{empty}\begin{center}  \Huge \bfseries}
	\renewcommand{\maketitlehookb}{\centering \Large  \textit{Vorlesungsmitschriften im Wintersemester 2018/19}}
\date{}



\begin{document}
	
\frontmatter
\maketitle
\begin{KeepFromToc}
	\tableofcontents
\end{KeepFromToc}
\chapter*{Vorwort}
Diese Vorlesungsmitschriften werden in der Vorlesung \textit{Einführung in die Algebra} von Prof. Jan Schröer im Wintersemester 2018/19 an der Universität Bonn angefertigt.\par
Wir versuchen, diese immer unter \url{https://pankratius.github.io} zu aktualisieren.

I will write them in English, as Prof. Schroer already provides a german version of his lecture notes. In addition, the first two lectures are ommited, as they were only motivational, but my motivation to draw a lot of pictures is fairly limited.

\mainmatter
\chapter{Gruppen}

\section{Definitionen - Grundlegendes}
\lipsum[1-5]
\section{Abbildungen von Gruppen}
\subsec{Gruppenhomomorphismen}
\lipsum[1]
\subsec{Gruppenisomorphismen}
\begin{lem}\label{thm:Apfel}
	Ich bin ein Apfel.
\end{lem}
\lec

\begin{defn}
	Ein \toidx{Gruppen\isom} ist ein Isomorphismus in der Kategorie der Gruppen. \Cref{thm:Apfel}
\end{defn}
\newpage
\section{Normalteiler und Quotientengruppen}
\begin{prop}
	Normalteiler sind ziemlich cool.
	\[
	\begin{tikzcd}
	G \ar{r}{\phi} \ar[swap]{d}{\pi} & H\\
	G/N \ar[swap]{ur}{\exists! f}
	\end{tikzcd}
	\]
	Wir erhalten für den Fall, dass $\phi$ surjektiv ist, folgendes Bild:

	\[
	\begin{tikzcd}
	G \ar[\surarr]{r}{\pi}  \ar[bend left = 25]{rrr}{\phi}& G/\ker\phi \ar{r}{\cong} & \im \phi \ar[\embarr]{r}{} & H
	\end{tikzcd}
	\]
\end{prop}
\lec
\thispagestyle{rlast}

\appendix
\chapter{Übungsaufgaben}
\section{some useful stuff from one of the exercise sheets}
\backmatter
\printindex

\end{document}